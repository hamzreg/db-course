\chapter{Конструкторский раздел}

\section{Проектирование базы данных}

\subsection{Таблицы базы данных}

База данных разрабатываемой информационной системы должна состоять из таблиц, содержащих данные о винах, поставщиках, покупателях, продажах, бонусных картах покупателей, их покупках и пользователях. Диаграмма, отражающая информацию о таблицах, показана на рисунке \ref{img:databaseEntities}.

\includeimage
    {databaseEntities}
    {f}
    {h}
    {1.0\textwidth}
    {Диаграмма базы данных}
    
Таблица Wine содержит сведения о винах и обладает полями, описанными в таблице \ref{tab:wine_structure} для структуры продукта виноторговли:
\begin{itemize}
	\item ID --- первичный ключ (целочисленный тип данных);
	\item Kind --- сорт вина (строковый тип данных);
	\item Color --- цвет вина (строковый тип данных), поле может хранить одно из трех значений: red, white, rose;
	\item Sugar --- сахар (строковый тип данных), поле может хранить одно из четырех значений: dry, semi-dry, semi-sweet, sweet;
	\item Volume --- объем вина (вещественный тип данных);
	\item Alcohol --- содержание алкоголя (вещественный тип данных);
	\item Aging --- выдержка (вещественный тип данных);
	\item Number --- число поставщиков, продающих данное вино.
\end{itemize}

Таблица Supplier хранит информацию о поставщиках и включает следующие поля:
\begin{itemize}
	\item ID --- первичный ключ (целочисленный тип данных);
	\item Name --- уникальное имя поставщика (строковый тип данных);
	\item Country --- страна производителя (строковый тип данных);
	\item Experience --- опыт поставки (вещественный тип данных);
	\item License --- права поставщика на добавление, удаление и редактирование товаров (логический тип данных);
\end{itemize}

Таблица SupplierWine реализует связь <<многие-ко-многим>> таблиц Wine и SupplierWine, поэтому обладает следующими полями:
\begin{itemize}
	\item ID --- первичный ключ (целочисленный тип данных);
	\item WineID --- ID вина, внешний ключ (целочисленный тип данных);
	\item SupplierID --- ID поставщика, внешний ключ (целочисленный тип данных);
	\item Price --- цена поставки (вещественный тип данных);
	\item Percent --- процент для формирования наценки (целочисленный тип данных);
	\item Rating --- рейтинг вина данного поставщика (вещественный тип данных).
\end{itemize}

Таблица Sale состоит из записей, содержащих сведения о продаже. Поля таблицы соответствуют описанным ранее в таблице \ref{tab:sale_structure} параметрам продажи:
\begin{itemize}
	\item ID --- первичный ключ (целочисленный тип данных);
	\item PurchaseID --- ID покупки, внешний ключ (целочисленный тип данных);
	\item SupplierWineID --- ID товара поставщика, внешний ключ (целочисленный тип данных);
	\item PurchasePrice --- закупочная цена (вещественный тип данных);
	\item Margin --- наценка (вещественный тип данных);
	\item Costs --- сумма издержек (вещественный тип данных);
	\item SellingPrice --- цена реализации (вещественный тип данных);
	\item Profit --- прибыль (вещественный тип данных);
	\item Date --- дата продажи (тип даты);
	\item WineNumber --- число продаваемого товара (целочисленный тип данных).
\end{itemize}

Таблица Purchase отображает информацию о покупках покупателя и имеет следующие поля:
\begin{itemize}
	\item ID --- первичный ключ (целочисленный тип данных);
	\item CustomerID --- ID покупателя, внешний ключ (целочисленный тип данных);
	\item Price --- цена покупки (вещественный тип данных);
	\item Status --- статус покупки (целочисленный тип данных).
\end{itemize}

Сведения о покупателях хранятся в таблице Customer со следующими полями:
\begin{itemize}
	\item ID --- первичный ключ (целочисленный тип данных);
	\item BonusCardID --- ID бонусной карты покупателя, внешний ключ (целочисленный тип данных);
	\item Name --- имя покупателя (строковый тип данных);
	\item Surname --- фамилия покупателя (строковый тип данных).
\end{itemize}

Таблица BonusCard содержит информацию о бонусных картах покупателя в следующих полях:
\begin{itemize}
	\item ID --- первичный ключ (целочисленный тип данных);
	\item Bonuses --- число бонусов, внешний ключ (целочисленный тип данных);
	\item Phone --- номер телефона (строковый тип данных).
\end{itemize}

Для хранения информации для авторизации используется таблица User со следующими полями:
\begin{itemize}
	\item ID --- первичный ключ (целочисленный тип данных);
	\item RoleID --- ID покупателя или поставщика, внешний ключ (целочисленный тип данных);
	\item Login --- уникальный логин пользователя (строковый тип данных);
	\item Password --- пароль пользователя (строковый тип данных);
	\item Role --- роль пользователя (строковый тип данных), поле может принимать одно из следующих значений: guest, admin, customer, supplier.
\end{itemize}

\subsection{Триггеры базы данных}

При удалении пользователя, который является покупателем, бонусная карта удаляемого покупателя также должна быть удалена. Для сохранения отчетности продаж запись о покупателе из таблицы покупателей не должна удаляться, но поле строки, указывающее на идентификатор бонусной карты должен быть равен NULL. Для обеспечения ссылочной целостности на таблице User должен быть определен DML-триггер BEFORE DELETE, схема алгоритма которого показана на рисунке \ref{img:delete_user_trigger}.

\includeimage
    {delete_user_trigger}
    {f}
    {h}
    {0.55\textwidth}
    {Схема алгоритма удаления пользователя}
    
При отмене покупки покупателе должна удалять запись соответствующей продажи, при этом поле покупки должно принимать значение, соответствующее статусу отмены. Для этого на таблице Sale должен быть определен DML-триггер BEFORE DELETE, схема алгоритма которого представлена на рисунке \ref{img:delete_sale_trigger}.

\includeimage
    {delete_sale_trigger}
    {f}
    {h}
    {0.4\textwidth}
    {Схема алгоритма удаления продажи}

\subsection{Роли базы данных}

Для поддержки безопасного проведения операций доступа, добавления, удаления и изменения данных в разрабатываемой базе данных должны быть созданы четыре роли:

\begin{enumerate}
	\item Неавторизованный пользователь должен иметь права на следующие действия:
		\begin{itemize}
			\item получение информации таблиц Wine и SupplierWine для просмотра информации о винах;
			\item добавление в таблицу User для регистрации в системе.
		\end{itemize}
	\item Покупатель должен иметь возможность на проведение следующих операций:
		\begin{itemize}
			\item получение сведений таблиц Wine, SupplierWine, Supplier, Purchase и BonusCard для просмотра информации о винах, поставщиках, покупках и бонусной карте;
			\item вставка в таблицы BonusCard для получения бонусной карты;
			\item добавление в таблицы Purchase и Sale для совершения покупки;
			\item удаление записей из таблицы Sale для отмены покупки;
			\item обновление таблицы Purchase при отмене покупки.
		\end{itemize}
	\item Поставщик должен обладать правами на выполнение следующих действий:
		\begin{itemize}
			\item получение данных таблиц Wine, SupplierWine и Sale для просмотра информации о винах и продажах;
			\item вставка в таблицы Wine и SupplierWine для добавления нового товара;
			\item изменение таблицы SupplierWine для редактирования товара;
			\item удаление из таблиц Wine и SupplierWine для удаления товара/
		\end{itemize}
	\item Администратор должен иметь возможность совершать все действия со всеми таблицами разрабатываемой базы данных.
\end{enumerate} 

\section{Проектирование информационной системы}

Для проектирования программного обеспечения был выбран объектно-ориентированный подход. Данное проектирование предполагает распределение обязанностей между компонентами разрабатываемого приложения. Поэтому информационная система будет разрабатываться как комплексный монолитный проект. Структура приложения должна состоять из трех уровней:

\begin{enumerate}
	\item Уровень бизнес-логики, на котором реализуются правила поведения объектов предметной области.
	\item Уровень доступа к данным, который включает в себя реализацию обращения к таблицам базы данных.
	\item Уровень представления, на котором создается пользовательский интерфейс.
\end{enumerate}

В описанной структуре запросы пользователя, принятые на уровне представления, передаются на уровень бизнес-логики, который обращается к уровню доступа к данным для обработки запросов. Схема проектируемой структуры показана на рисунке \ref{img:structure}.

\includeimage
    {structure}
    {f}
    {h}
    {0.8\textwidth}
    {Схема структуры приложения}

\section*{Вывод}

В данном разделе было проведено проектирование базы данных: представлены описание таблиц разрабатываемой базы данных, алгоритмы работы триггеров и права различных ролей пользователей. Кроме того, была описана структура разрабатываемого программного обеспечения.