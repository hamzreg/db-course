% Языки
\usepackage[english,russian]{babel}


% Поля
\usepackage{geometry}
\geometry{right=10mm}
\geometry{left=30mm}
\geometry{top=20mm}
\geometry{bottom=20mm}


% Заголовки
\usepackage{titlesec}

\titleformat{\section}{\normalsize\bfseries}
{\thesection}{1em}{}

\titlespacing*{\chapter}{0pt}{-30pt}{8pt}
\titlespacing*{\section}{\parindent}{*4}{*4}
\titlespacing*{\subsection}{\parindent}{*4}{*4}

\titleformat{\chapter}{\LARGE\bfseries}{\thechapter}{20pt}{\LARGE\bfseries}
\titleformat{\section}{\Large\bfseries}{\thesection}{20pt}{\Large\bfseries}


% Интервал

% Междустрочный
\usepackage{setspace}
\onehalfspacing

% Абзацный отступ
\usepackage{indentfirst}
\setlength{\parindent}{1.25cm}
\linespread{1.25}

% Подписи
\usepackage{caption}
\captionsetup{labelsep=endash} % Объект - описание

% Рисунки
\usepackage{caption}
\captionsetup[figure]{name={Рисунок}} % Рис. -> Рисунок

\usepackage{svg}
\usepackage{float}
\usepackage[justification=centering]{caption}
\usepackage{pgfplots}
\pgfplotsset{compat=1.9}
\usetikzlibrary{datavisualization}
\usetikzlibrary{datavisualization.formats.functions}
\usepackage{graphicx}
\newcommand{\img}[3] {
    \begin{figure}[h]
        \center{\includegraphics[height=#1]{assets/img/#2}}
        \caption{#3}
        \label{img:#2}
    \end{figure}
}
\newcommand{\boximg}[3] {
    \begin{figure}[h]
        \center{\fbox{\includegraphics[height=#1]{assets/img/#2}}}
        \caption{#3}
        \label{img:#2}
    \end{figure}
}


% Код
\usepackage{listings}
\usepackage{xcolor}

\lstset{ %
	language=python,   					% выбор языка для подсветки	
	basicstyle=\small\sffamily,			% размер и начертание шрифта для подсветки кода
	numbers=left,						% где поставить нумерацию строк (слева\справа)
	%numberstyle=,					% размер шрифта для номеров строк
	stepnumber=1,						% размер шага между двумя номерами строк
	numbersep=5pt,						% как далеко отстоят номера строк от подсвечиваемого кода
	frame=single,						% рисовать рамку вокруг кода
	tabsize=4,							% размер табуляции по умолчанию равен 4 пробелам
	captionpos=t,						% позиция заголовка вверху [t] или внизу [b]
	breaklines=true,					
	breakatwhitespace=true,				% переносить строки только если есть пробел
	escapeinside={\#*}{*)},				% если нужно добавить комментарии в коде
	backgroundcolor=\color{white},
}


% pdf
\usepackage{pdfpages}
\usepackage[justification=centering]{caption} % Настройка подписей float объектов


% Нумерация страниц
%\setcounter{page}{0}


% Отточие
\usepackage{tocloft}
\renewcommand{\cftchapleader}{\cftdotfill{\cftdotsep}}


% Для описания множеств
\usepackage{amsfonts}